\documentclass[a4paper]{article}
\title{Praktikrapport - Universal Robots}
\author
{
    Johannes Ehlers Nyholm Thomsen\\ 
    \texttt{joha321j@edu.ucl.dk}
}

% Danish symbols.
\usepackage[utf8]{inputenc}
\usepackage[danish]{babel}
\usepackage[T1]{fontenc}

%Links and clickable ToC
\usepackage{hyperref}

\begin{document}

\maketitle

\newpage

\tableofcontents

\newpage

\section{Indledning}

\section{Forventninger}
Som led i datamatikeruddannelsen var jeg i et praktikforløb hos Bankdata.
I det praktikforløb blev jeg en del af et udviklingsteam,
hvor jeg hjalp dem med deres projekter.

Inden praktikforløbet hos Universal Robots(UR) startede, havde jeg en forventing om,
at det ville forløbe på samme måde.
Jeg så frem til at arbejde i et område,
der ikke var lige så konservativt som finansverdenen.
Jeg havde specifikt gået efter et firma,
der arbejdede i en anden sektor end Bankdata for at få et bredere syn på,
hvad jeg ville kunne arbejde med, når jeg var færdiguddannet.

\subsection{Læringsmål}
Jeg har haft opstillet nogle mål for, 
hvad jeg gerne ville have ud af mit praktikforløb hos UR.
Målene har hjulpet mig med at klargøre over for mig selv,
hvad jeg har af forventninger til mig selv under praktikforløbet.

I store træk gik læringsmålene ud på, 
at jeg gerne ville blive bedre til at være en del af et selvstyrende
udvilkingsteam.

\subsubsection*{Personlige Læringsmål}
\begin{itemize}
    \item Få mere erfaring i at indgå fagligt og professionelt i en 
    softwareudviklingsvirksomhed
    \item Forbedre mine evner i at deltage i et professionelt selvstyrende 
    udviklingsteam
    \item Få erfaring med brug af \href{https://en.wikipedia.org/wiki/OKR}{OKR's}
     til at opsætte og organisere mit teams mål for vores produkt.
\end{itemize}

\subsubsection*{Tekniske Læringsmål}
Mine tekniske læringsmål er opstillet på baggrund af en liste af teknologier
som teamet, jeg ville blive en del af, brugte.

\begin{itemize}
    \item Forbedre mine evner som full stack udvikler i .NET teknologi stacken
    \begin{itemize}
        \item Rest API udvikling med ASP.NET Core
        \item Single page application(SPA) udvilking med Blazor
        \item Brug af Entity Framework som Object Relational Mapper(ORM)
    \end{itemize}
    \item Forbedre mine evner inden for OpenAPI dokumentation ved brug af
    Swagger
    \item Få erfaring og kompetencer inden for brug af ElasticSearch
    \item Få erfaring inden for brug af MongoDB
\end{itemize}

Som alle ved overlever en god plan ikke længere end første møde med fjenden,
og jeg har derfor ikke opnået alle de læringsmål, som jeg har beskrevet her.
Se afsnit \ref{praktikforloeb} for en forklaring på hvorfor.
Dog har jeg lært en helt masse andet, 
som jeg vil komme ind på i afsnit \ref{udbytte}.

\section{Praktikforløbet}
\label{praktikforloeb}
I dette afsnit vil jeg beskrive selve praktikforløbet.
Først vil jeg beskrive, hvordan den daglige organisering var.
Dernæst vil jeg komme med eksempler på arbejdsopgaver.

\subsection{Ejet produktteam}
På første dag i praktikken blev jeg indformeret om,
at jeg ikke ville blive en del af myUR - Fleet.
myUR - Fleet er det produktteam, som jeg fik at vide under praktikinterviewet,
at jeg ville blive en udvikler hos.

I stedet fik Kasper, Mike og jeg at vide,
at vi 3 ville lave vores ejet produktteam.
Dernæst fik vi en problemstilling, som vi skulle løse.

\subsubsection{Problemstilling}
Universal Robots er inden for det sidste år skiftet struktur til at være 
produkt-orienterede i stedet for projektorienterede.
I den forbindelse har de forsøgt at give deres teams mere selvstyre ved at 
adoptere OKR's.

Kort fortalt går OKR's ud på, 
at den øverste ledelse sætter overordnede mål for virksomheden.
Ud fra de mål sætter afdelingerne mål for,
hvordan de kan hjælpe ledelsen med at opnå deres mål.
Til sidst sætter produktteamsne mål for, 
hvordan de kan hjælpe deres afdeling med at opnå afdelingens mål.
På denne måde har produktteamet selv ansvar for,
hvordan de giver værdi til virksomheden.

En del af OKR's er key results. 
Key results er målbare resultater, der skal bruges til at se,
om produktteamet er på vej mod deres mål.

Problemet var, at der var ingen god måde at visualisere 
fremskridtene for produtkteamsne.
Derfor var det svært for teamsne og se, hvordan det går med at opnå deres mål.

\subsubsection{Løsning}
I samarbejde med vores mentor, Julian, blev vi enige om, at vi skulle lave en 
løsning, der kunne tage i mod data og visualisere det i form af 
grafer, tabeller og lign.

\subsubsection{Organisering}
Julian og hans kollega Dennis ville være vores produkt owners,
mens vores chef Mark ville være vores stakeholder.
Ellers var Kasper, Mike og jeg fuldstændigt selvstyrende.

Vi fik et confluence rum, som vi har brugt til at dele forskellige filer, 
domæne modeller og wireframes blandt andet.

Til den daglige styring og organisering af vores arbejde,
fik vi en azure devops portal.

\subsection{Full Stack udvilking}

\section{Udbytte}
\label{udbytte}
\subsection{Selvstyring}
\subsection{Ny Teknologi}
\subsubsection{Integration med Grafana}

\end{document}