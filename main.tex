\documentclass[a4paper]{article}
\title{Praktikrapport - Universal Robots}
\author
{
    Johannes Ehlers Nyholm Thomsen\\ 
    \texttt{joha321j@edu.ucl.dk}
}

% Danish symbols.
\usepackage[utf8]{inputenc}
\usepackage[danish]{babel}
\usepackage[T1]{fontenc}

%Links and clickable ToC
\usepackage{hyperref}

\begin{document}

\maketitle

\newpage

\tableofcontents

\newpage

\section{Forventningerne}
\subsection{Hvad ville jeg gerne lære}

\subsubsection*{Personlige Læringsmål}
\begin{itemize}
    \item Få mere erfaring i at indgå fagligt og professionelt i en 
    softwareudviklingsvirksomhed
    \item Forbedre mine evner i at deltage i et professionelt selvstyrende 
    udviklingsteam
    \item Få erfaring med brug af \href{https://en.wikipedia.org/wiki/OKR}{OKR's}
     til at opsætte og organisere mit teams mål for vores produkt.
\end{itemize}

\subsubsection*{Tekniske Læringsmål}
\begin{itemize}
    \item Forbedre mine evner som full stack udvikler i .NET teknologi stacken
    \begin{itemize}
        \item Rest API udvikling med ASP.NET Core
        \item Single page application(SPA) udvilking med Blazor
        \item Brug af Entity Framework som Object Relational Mapper(ORM)
    \end{itemize}
    \item Forbedre mine evner inden for OpenAPI dokumentation ved brug af
    Swagger
    \item Få erfaring og kompetencer inden for brug af Grafanaß
    \item Få erfaring inden for brug af PostgreSQL
\end{itemize}

\section{Praktikforløbet}
I dette afsnit vil jeg beskrive selve praktikforløbet.
Først vil jeg beskrive, hvordan den daglige organisering var.
Dernæst vil jeg komme med eksempler på arbejdsopgaver.

\subsection{Ejet produktteam}
På første dag i praktikken blev jeg indformeret om,
at jeg ikke ville blive en del af myUR - Fleet.
myUR - Fleet er det produktteam, som jeg fik at vide under praktikinterviewet,
at jeg ville blive en udvikler hos.

I stedet fik Kasper, Mike og jeg at vide,
at vi 3 ville lave vores ejet produktteam.
Dernæst fik vi en problemstilling, som vi skulle løse.

\subsubsection{Problemstilling}
Universal Robots er inden for det sidste år skiftet struktur til at være 
produktorienterede i stedet for projektorienterede.
I den forbindelse har de forsøgt at give deres teams mere selvstyre ved at 
adoptere OKR's.

Kort fortalt går OKR's ud på, 
at den øverste ledelse sætter overordnede mål for virksomheden.
Ud fra de mål sætter afdelingerne mål for,
hvordan de kan hjælpe ledelsen med at opnå deres mål.
Til sidst sætter produktteamsne mål for, 
hvordan de kan hjælpe deres afdeling med at opnå afdelingens mål.
På denne måde har produktteamet selv ansvar for,
hvordan de giver værdi til virksomheden.

En del af OKR's er key results. 
Key results er målbare resultater, der skal bruges til at se,
om produktteamet er på vej mod deres mål.

Problemet var, at der var ingen god måde at visualisere 
fremskridtene for produtkteamsne.

\subsubsection{Løsning}
I samarbejde med vores mentor, Julian, blev vi enige om, at vi skulle lave en 
løsning, der kunne tage i mod data og visualisere det i form af 
grafer, tabeller og lign.


\subsubsection{Organisering}
Julian og hans kollega Dennis ville være vores produkt owners,
mens vores chef Mark ville være vores stakeholder.

Derudover var Kasper, Mike og jeg fuldstændigt selvstyrende.


\subsection{Full Stack udvilking}

\section{Udbytte}
\subsection{Selvstyring}
\subsection{Ny Teknologi}
\subsubsection{Integration med Grafana}

\end{document}